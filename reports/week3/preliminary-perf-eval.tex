\section{Preliminary Performance Evaluation}
\label{sec:perf-eval}

\begin{table}[ht]
\centering
\caption{Confusion matrix of our classifier}
\begin{tabular}{lccccc}
\toprule
 & Sitting & Walking & Running & Stairs up & Stairs down\\
\midrule
Sitting & 0.75 &  &  &  0.14 & 0.10\\
Walking &  &  &  & 0.78 & 0.22\\
Running &  & 0.1 & 0.9 &  & \\
Stairs up &  &  &  & 1.00 & \\
Stairs down &  & 0.3 &  & 0.4 & \\
\bottomrule
\end{tabular}
\label{tab:confusion-matrix}
\end{table}


Our classifier as of now performs well on running, but does not differentiate well between Walking, Stairs up and Stairs down. Specifically there is a heavy bias toward Stairs up.

We might be affected by the ``Curse of Dimensionality`` \cite{wikipedia2} which means that the required size of the training data grows exponentially with the dimension of the feature vector because the volume becomes more sparse. As we have 9 features/dimensions this might be our problem.

Possible solutions to improve our classifier accuracy might be removing or reducing our features/dimensions. Removing outliers (noise) in our training data might also help \cite{wikipedia1}.
Another possible major improvement we plan to make is to use Mahalanobis distance or some other distance instead of Eucledian distance, as Eucledian distance does not perform well with high-dimensionality data \cite{wikipedia1}.