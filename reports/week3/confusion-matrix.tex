\section{Confusion Matrix}
\label{sec:confusion-matrix}
In order to create the confusion matrix, we first trained our classifier --- which uses the k-nn algorithm --- by gathering about 110 samples for each activity in the \texttt{TrainFragment} of our Android application. After gathering the 550 samples, another fragment, the \texttt{TestFragment}, allows the user to choose between taking 1, 5, 10, or 50 measurements to test the accuracy of the classifier. In this fragment the user also needs to select the activity that is he or she is going to perform before one can start taking the measurements. This is needed so that the application can write both the actual activity as well as the expected activity to the \texttt{testing.csv} file.

Each line in the \texttt{testing.csv} file contains three comma-separated values: the timestamp at which the measurement was taken, the activity as measured by the classifier, and the expected activity (the activity that the user was performing). By processing this data we can construct the confusion matrix shown in \autoref{tab:confusion-matrix}.

\begin{table}[ht]
\centering
\caption{Accuracy of our k-nn classifier using 550 samples.}
\begin{tabular}{lccccc}
\toprule
 & Sitting & Walking & Running & Stairs up & Stairs down\\
\midrule
Sitting & 0.75 &  &  &  0.14 & 0.10\\
Walking &  &  &  & 0.78 & 0.22\\
Running &  & 0.1 & 0.9 &  & \\
Stairs up &  &  &  & 1 & \\
Stairs down &  & 0.3 &  & 0.4 & \\
\bottomrule
\end{tabular}
\label{tab:confusion-matrix}
\end{table}
